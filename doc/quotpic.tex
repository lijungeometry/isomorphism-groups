%-----------------------------------------------------------------------------
% Beginning of quotpic.tex
%-----------------------------------------------------------------------------
%
% AMS-TeX 2.0 file for DIMACS volumes.
%
\input amstex
\documentstyle{dimacs}
\NoBlackBoxes
\leftheadtext{DEREK F. HOLT AND SARAH REES}
\rightheadtext{A GRAPHICS SYSTEM FOR FINITELY PRESENTED GROUPS}
\NoBlackBoxes

\topmatter
\title 
A Graphics System for Displaying Finite
Quotients of Finitely Presented Groups
\endtitle
\author Derek F. Holt and Sarah Rees\endauthor
\address 
Mathematics Institute,
University of Warwick,
Coventry CV4 7AL,
U.K.
\endaddress  %address for author one
\email dfh\@maths.warwick.ac.uk\endemail
\address
Mathematics Department,
University of Newcastle,
Newcastle NE1 7RU,
U.K.
\endaddress %address for author two
\email sarah.rees\@newcastle.ac.uk\endemail

%  The following items give publication information for the DIMACS logo
\cvol{00}
\cvolyear{0000}
\cyear{0000}

%  Math Subject Classifications 
\subjclass Primary 20-04, 20F05;\endsubjclass
\abstract
We describe a graphics package that the authors are currently developing for
plotting the finite quotient groups of a given finitely presented group.
For the graphics display, an X-Windows server is required, which may be
coloured or black and white.
\endabstract

\thanks This paper is in final form and no version of it will be submitted for
publication elsewhere\endthanks

\endtopmatter

\document

\head 1. Introduction                % bold, centered; 
\endhead                             % don't type final punctuation 
In this paper, we describe an interactive graphics package that the authors are
currently developing for plotting the finite quotient groups of a given
finitely presented group $G$. For the graphics display, an X-windows server is
required, which may be coloured or black and white. The user must first
make a file containing the given presentation of $G$; thereafter, all
computations are done interactively, by using a mouse to select menu options.
A finite quotient $Q = G/M$ of $G$ is represented in the
display by a vertex, which can be thought of as being labelled by the normal
subgroup $M$. Given two such vertices $M$ and $N$ such that $|G/N|$ has
more prime factors (counted with repetition) than $|G/M|$,
$N$ will always appear lower than $M$. If $M$ is joined to $N$ either
by an edge, or by a sequence of edges passing through intermediate vertices,
then the inclusion $M \supset N$ holds.
Initially, there will be a single vertex $G$ corresponding to the trivial
quotient; further vertices are plotted as the corresponding quotients are
calculated. The system can output a postscript file for producing a good
hard copy of the lattice that has been plotted.

We did not develop the system with any particular classes of groups in mind.
However, the order of the quotients that we can plot is highly dependent on the
particular example, and on the structure of the quotient itself.
Quotients that are
extensions of nilpotent groups by groups of small order can often be computed
(using the $p$-quotient algorithm) up to a very large order (such as
$p^{1000}$ for a prime $p$, where can be large). 	
On the other hand, in some examples, such as free groups, and
groups with a large number of generators, it might be impractical to plot all
quotients even up to index 20 or 30, since there could be far too many of them.
The advantage of the
interactive approach is that we are able to explore the quotient structure
in those directions in which we are interested, and we are never obliged to
plot more of them than we want to. In order to have some sort of
specific goal in mind, we decided to provide at least
potential access to all quotients up to order 10000.
To justify this, we prove the following easy result.
\proclaim {Proposition}
Let $G$ be a group of order at most \rom {10000}. Then $G$ has a normal soluble
subgroup $N$, such that $S \subseteq G/N \subseteq $ Aut$(S)$, where $S$ is
either trivial, or a nonabelian simple group, or isomorphic to $A_5 \times
A_5$.
\endproclaim
\demo{Proof}
Let $N$ be the maximal normal soluble subgroup of $G$. Then either $N = G$, or
$G/N$ has a minimal normal subgroup $M/N$ which is a direct product of
isomorphic finite nonabelian simple groups. Since $|G| \leq 10000$,
$M/N$ must be either simple or isomorphic to $A_5 \times A_5$. In either case,
$M/N$ is self-centralizing in $G/N$, and so $G/N \subseteq $ Aut$(M/N)$.
\enddemo

It is clear that this proposition can easily be made valid up to much larger
orders by making obvious additions to the list of possibilities for $S$. 
Guided by this result, we have provided our system with two principal
facilities. The first is to find the epimorphisms from $G$ onto the
relevant subgroups of Aut$(S)$, for all appropriate groups $S$. This is done
interactively by the user, by selecting the group $S$ from a list on a menu.
The second facility can be described as follows. Given a finite quotient
$G/M$ of $G$ of sufficiently small order (at most 5000 will do in this case),
find those quotients $G/N$ of $G$ with $M \supset N$ and $M/N$ an elementary
abelian $p$-group for some prime $p$. If we can do this, then it is clear from
the proposition that we can, in principle, find all quotients of $G$ up to the
required order. We achieve this as follows. First, we use an abelianized
Reidemeister-Schreier process (as originally programmed by Havas in \cite {1})
to compute the largest elementary abelian
$p$-quotient $M/M^{(p)}$ of $M$ for a particular prime $p$ (which the user can
select from the menu). Then $M/M^{(p)}$ can be regarded as a $K(G/M)$-module,
where $K$ is the finite field of order $p$, and the normal subgroups $N$ that
we are seeking correspond precisely to the $K(G/M)$-submodules $N/M^{(p)}$ of
$M/M^{(p)}$.  In fact there was already in existence an efficient collection
of programs for computing the submodule lattice of a module over a prime
field, based on the well-known `Meataxe' algorithm of Richard Parker (see
\cite{10}). This had been written at Aachen, and we are grateful to the
Aachen group for allowing us to incorporate these programs into our system, 
and thereby to complete our requirements (see \cite{8}).

In addition to these basic facilities, we have provided a number of other
options, which experience has suggested are useful in examples which arise
in practice. For example, when $G/M$ is not too large, we can compute the
complete set of abelian invariants of $M/[M,M]$, and thereby determine
which prime numbers $p$ are relevant. This will also sometimes reveal that
$M$ and hence $G$ is infinite. We have also provided a version of the
Canberra $p$-quotient algorithm (until recently, known as the NQA; see, 
for example,
\cite{3}) to compute the quotients of the lower exponent-$p$-central series of
$M$.  The basic lattice operations intersection and join on the vertices
(thought of as normal subgroups of $G$) are also available; indeed, these can
be used to ascertain whether or not one vertex is contained in another when
this is not already clear. Although we are not providing sufficient facilities
to completely identify a given quotient up to isomorphism, this is possible in
many cases. As an easy aid to this, it is possible to count the numbers of
elements of each order in a quotient. This is particularly useful if we are
trying to distinguish one group from another. The simplest example is the
use of this technique to distinguish between the dihedral and quaternion
groups of order 8.

Of course there is a considerable overlap between these facilities and those
provided by the system SPAS developed in Aachen, and for some particular
computations that stretch resources to their limits, the best results will
be obtainable by using SPAS followed by another program, such as the integral
matrix diagonalization algorithm, as implemented by Havas and Stirling
(see \cite{4}).
Although we hope to improve some of our particular implementations in the
future (or incorporate existing implementations), our principal aim has been
not so much to push back barriers of effective computability as to provide
a convenient and user-friendly interface to this particular problem, which
will hopefully enable the nonexpert user to gain a rapid overview of the
finite quotient structure of a finitely presented group.
We have also been guided by the aim of providing the theoretical
capability of finding all quotients of $G$ up to a given order. Unlike
in SPAS, we do not make use of the low-index subgroup algorithm (see, for
example, Section 6 of \cite{9}) to find quotients that have permutation
representations of small degree. However, we may well decide in the future
to incorporate this, since it can sometimes lead to more rapid determination
of particular quotients.

The design of our system is that
we have a control program {\bf quotpic} which manages the graphics display
and remembers details of the part of the lattice that has been computed so far.
This program calls up other programs to perform the individual computations
required. These other programs input and output from and to files, which are
read by {\bf quotpic}. The user can of course run these programs
individually if that should be more convenient, or copy and edit the files
for use by other programs, but for most of the time, even for an expert user,
it will be more convenient to work from within {\bf quotpic}.

An essential part of the design of the system
is that a quotient $G/M$ of sufficiently small order will be stored in a file
as the regular permutation representation (or equivalently, the Cayley graph) of
$G/M$. It is for such subgroups $M$ that it is possible to carry out further
computations on $M$ in order to find larger quotients $G/N$ with $M \supset N$.
All of the programs that construct such $G/N$ are capable of computing the
regular permutation representation of $G/N$ provided that the order is
sufficiently small. (This is  set by default to 10000, but it can easily be
increased up to at least 100000.) For quotients $G/N$ that are too large,
this is not possible. We have therefore distinguished between three types
of vertices, the green (or circular, for those with a black and white screen),
the yellow (or triangular) and the red (or octagonal). 
The green vertices are those for which a regular
permutation representation is stored. For the yellow (or triangular) vertices,
a faithful but not regular permutation representation of the quotient is stored,
and for the red no permutation representation is stored. Fewer functions
are available for yellow and red vertices than for green vertices.

The remainder of this paper is organized as follows. In the second section,
there is a more detailed description of the control program {\bf quotpic}. In
the third section, we deal with the individual programs that are called up
by {\bf quotpic}. Several of these were written in connection with an earlier
project of the authors to attempt to decide whether two finitely presented
groups are isomorphic or not (see \cite{7}), and so they have already been
described and do not require so much detail here. In the fourth
section, we describe some results and present some illustrations.
Finally, in the fifth we outline our plans for the future development of the
program.

We are grateful to Michael Ringe in Aachen for providing us with the code
for the submodule lattice package, and general assistance in its use,
and to Steve Rumsby at Warwick for help with programming the X-windows material.
Helpful suggestions that have influenced the design of the system have been
received from various people, but particularly from
George Havas, Joachim Neub\"{u}ser, Mike Newman and Eamonn O'Brien.

\head 2. The X-Window program {\it quotpic}                % bold, centered; 
\endhead                             % don't type final punctuation 
The program {\bf quotpic} starts by opening a window on the screen, which contains a
green circular vertex corresponding to the trivial quotient of $G$. A click
with the left mouse button opens a menu of available functions, and the user 
can then start to  construct the lattice from the top down.

As we have already mentioned, {\bf quotpic} constructs vertices of three 
different 
types, depending on the amount of information that is held by the program on 
the corresponding quotient. Different functions are available at the different 
types of vertices. Whatever the type of the vertex, the menu listing the 
functions with which the lattice can be built downward from that vertex is
always opened by a click with the left mouse button near to the vertex. A 
further click in the menu selects a function, and  causes a parameter menu to
open if the function requires input parameters. For almost all functions a 
time parameter limits the amount of time that the program will spend on the 
function (if time runs out, the program will simply terminate the calculation
and wait for the user to make another choice). Many functions require
a prime parameter to be set; sometimes the user needs to select an exponent,
an exponent-$p$ class (i.e. length of the lower exponent-$p$-central series, see
\cite{3}) or a permutation group. All such parameters are selected from
the parameter menu. 

Clicking on the right hand mouse button causes a menu to be opened which 
allows the selection of general `housekeeping' tasks.  Vertices and edges can
be deleted, the lattice can be resized or refreshed, hard copies can be taken 
of the lattice in its current state (a postscript file from which
a hard copy can be printed is always created on exit from the program, but 
sometimes it is also useful to have a hard copy of intermediate stages of the 
calculation), the global parameter which controls the maximal degree for
which the program would attempt to construct a regular permutation 
representation can be reset, and the program can be terminated. 

Clicking on the middle mouse 
button allows the repositioning of vertices. The program automatically
plots  vertices in such a way that  quotients of the same order are plotted
along a horizontal line. In fact, if we define the {\it level} of a quotient
to be the number of primes dividing its order (counted with repetitions), then,
by default, all quotients with the same level are plotted on the same
horizontal line. Quotients of the same level but differing orders may be 
separated vertically using the mouse, but vertices of the same 
orders may not; thus, if a vertex is moved vertically, all vertices of
the same order as it move with it. It is not possible to move a vertex
to a position above vertices of a smaller level than itself.

Information about the results of a function call is given in a special
output window which opens up as necessary at the bottom of the main program 
window. The user will be informed if the function was unable to complete
for some reason, such as lack of available time, or memory (lack of
memory does not cause {\bf quotpic} to exit), or if the result of
the calculation was trivial. Also the user will be given information in this 
window such as how many quotients of the group have been found isomorphic to a
particular group in the program's list, or how many intermediate
quotients there are between two previously specified ones (see (e) below),
and will then be
invited to choose how many of these to plot. Small message windows issue running
instructions to the user (e.g. to wait, or select a vertex or edge with the
mouse).

The full range of functions is available at those (green, circular) vertices 
corresponding to quotients $G/M$ for which the program holds a regular 
permutation representation; that is, functions to
\roster
\item"(a)" calculate the abelian invariants of $M/[M,M]$, 
\item"(b)" extend downwards by various abelian sections, 
\item"(c)" calculate the sequence of orders of elements in $G/M$, 
\item"(d)" calculate a presentation of $M$ (which is then output to a file), 
\item"(e)" calculate (and plot as required) all normal subgroups between $M$ 
and some normal $N \subset M$, where $M/N$ is elementary abelian (and 
either maximal elementary abelian or a section in a
lower exponent-$p$-central series),
\item"(f)" calculate intersections and joins of $M$ with a second normal 
subgroup (in general, but not always one for which some, not necessarily 
regular, permutation representation is stored). 
\endroster
If $G=M$ we are at the top of the 
lattice, and then have
the facility, already mentioned, to find all quotients of $G$ isomorphic to a
specified  permutation group in our list, and then plot as many of the
corresponding new vertices as are required by the user.
Of these facilities, the functions to construct intersections and joins are
also available at the (yellow, triangular) vertices for which the program 
holds non-regular
permutation representations, and in exceptional situations even at
the (red, octagonal) vertices for which no permutation information is held.
Provided the order of a quotient is less than the maximum allowed degree for a
regular permutation representation, the user has the choice of 
constructing a regular 
representation  for a yellow vertex, thus turning it into a green vertex, 
equipped with the full choice of functions.
Edges are also divided into categories, according to their origins,
and the different categories of edges are plotted with different
colours on a colour screen. It is actually the category of the edge joining
vertices $G/M$ and $G/N$ which determines whether or not  the intermediate
normal subgroups can be found; thus this facility is potentially available at
any $G/M$, whatever the type of the vertex.
 
As a new quotient $G/N$ is built from an existing one $G/M$, provided the
order of $G/N$ is less than 1000, a check is carried out to see whether or
not the quotient has already been constructed in some other way and, if it
has, the appropriate edges are drawn to the existing vertex, and no new one 
is constructed. For higher indices however this check is not carried out
(because it can be somewhat slow), and it is up to the user to manually
check for equality between appropriate quotients using the intersection 
function. A message window indicates to the user that the automatic 
equality check has not been carried out.

The program does not automatically search for all possible inclusions between
normal subgroups as it constructs them. This would also be a lengthy business.
The user can verify a suspected inclusion using the intersection function.
 

\head 3. The programs called by {\it quotpic}                % bold, centered; 
\endhead                             % don't type final punctuation 
As was mentioned in the Introduction, the first major facility required is
to compute the epimorphisms from our finitely presented group $G$ onto groups
$H$ satisfying $S \subseteq H \subseteq$Aut$(S)$, where $S$ is a member of a
given list of groups. (Since we are interested in the quotients of $G$, we
regard two such epimorphisms as being equivalent if they have the same kernel
or, equivalently, if one is equal to the other followed by an automorphism of
$H$.) This is done using our program {\bf permim}. Since {\bf permim } has
already been described in some detail in \cite{5} and \cite{7} we shall not
repeat this here. The relevant groups $H$ are stored as permutation groups in
a library of files, which contain bases and strong generating sets for the
groups and also representatives of their conjugacy classes. The user has to
select the group $S$, and then all epimorphisms onto all groups $H$ between
$S$ and Aut$(S)$ are sought. Since this number can be very large in some
examples, the user is informed at the end of the calculation how many of
them there are onto each of these intermediate groups, and must then decide
how many of them, if any, to plot on the screen. As
with all of the programs that are called by {\bf quotpic}, it is also possible
(indeed mandatory) for the user to impose an upper cpu-time limit on the
particular process, and it will abort with a message if this limit is
exceeded. It is a fairly easy matter for the authors to add additional groups
$S$ to the list if required, but this normally requires doing some
computations (of conjugacy classes, for example) outside of the system
(usually in CAYLEY), and then converting the output to our file format.

The second major facility is provided by the programs that use the
Reide\-meister-Schreier algorithm for computing presentations of a subgroup $H$
of $G$ (see \cite{1} for a description of the first computer implementation
of this process).
In order to describe these, we have to repeat a small amount of the
material in \cite{7}.
Let $P$ be a transitive permutation group on
the finite set $\Omega = \{1,2, \ldots , t\}$,  and let
$\phi: G \longrightarrow P$ be an
epimorphism, where
$G = \langle g_1, g_2, \ldots , g_m | r_1, r_2, \ldots , r_s \rangle$.
We shall regard the generators $g_i$ as acting on $\Omega$
via $\phi$.  Let $Q = G_1$ be the stabilizer of
the point  1  in $P$,  and let $H = \phi^{-1}(Q)$.
Currently, in the applications to {\bf quotpic}, $H$ is always a normal
subgroup of $G$ and so $P$ is a regular permutation group, but this
is not a necessary assumption for most of what follows
(and in future developments to {\bf quotpic} we shall probably relax this
restriction).
For proofs see, for example, Section 4 of \cite{9},
bearing in mind that $P$ can be regarded as a complete coset table for
$G$ over the subgroup $H$.
For each $i$ in $\Omega - \{1\}$,  we can
regard one particular equation $j^{g_k} = i$ as being the definition of  $i$,
where $j$ is already defined,  and  1  is regarded as being defined initially.
This associates an element $g(i)$ of $G$  to each point
$i$ in $\Omega$, with the property that $1^{g(i)}=i$, where $g(1)=1$ and
$g(i)=g(j)g_k$ with $j^{g_k}=i$ a definition. The transversal
$\{ g(i) | 1 \leq i \leq t \}$ is therefore a Schreier transversal as defined
in \cite{9}.
To each equation  $j^{g_k} = i$ that is not a definition,
we can define an element $h(j,k)$ of $H$ by $g(j)g_k = h(j,k)g(i)$. The $h(j,k)$
then generate $H$.
We obtain the relators for $H$  as follows. For each relator $r_k$ of $G$ and
each point $i$ in $\Omega$, we can use the equations $g(j)g_k = h(j,k)g(i)$
to obtain an equation $g(i)r_k = w(i,k)g(i)$,  where $w(i,k)$ is a word in
the $h(j,k)$.  Then the $w(i,k)$ form a set of defining relators for $H$
with respect to the generators $h(j,k)$.

There are now several different ways in which we can proceed, depending on
what precisely is required. If we want to compute the
abelian quotient invariants of $H$, we apply
the standard integer diagonalization algorithm to the presentation, although it is important to
check for integer overflow, since this becomes a serious danger when the
index of $H$ in $G$ grows beyond a few hundred. At some stage, we hope to
install a better implementation that either uses arbitrarily large integers
or uses modular techniques as described in \cite{4}. To avoid this danger,
it is possible for the user to choose an integer $m > 0$ and compute the
largest abelian quotient $H/H^{(m)}$ of $H$ of exponent dividing $m$. The case
in which $m$ is a prime $p$ has been implemented separately, since it can be
done much more efficiently than the general case.
This calculation is feasible for subgroups $H$ of much larger index - in fact,
it has completed successfully for indices as high as 50000.
There is also an option to compute the largest such quotient $H/H^{(p)}[G,H]$
that is centralized by the conjugation action of $G$.

Sometimes we simply want a presentation of the kernel.  (We need that for
instance if we are going to apply the $p$-quotient algorithm.) This is
technically more complicated, because we can no longer assume throughout that
the generators $h(j,k)$ of $H$ commute with each other. Instead, we carry out
a restricted set of Tietze transformations to remove redundant relators and
simplify a relator if some cyclic conjugate of it has more than half of its
length in common with another relator. We only eliminate generators, however,
when they are either trivial or actually equal to another generator.  The
reasons for this are discussed in the next paragraph.

Assume now that $H$ is normal in $G$ and that we have used one of the above
methods to compute a quotient $H/K$ of $H$. Notice that, in each of the cases
described above, $K$ is characteristic in $H$ and hence normal in $G$.
Furthermore, we have a definition of $H/K$ (such as a power-commutator
presentation) in which we can compute efficiently, together with an explicit
epimorphism $\psi $ from  $H$ to  $H/K$, giving the images of the
generators $h(j,k)$ of $H$. The elements of $G/K$ have
the form  $\overline {h}g(j)$,  for $\overline{h}$ in $H/K$ and
$j = 1,2, \ldots ,t$. Since
$\overline{h}g(j)g_k = \overline{h} \psi(h(j,k))g(i)$,
this enables us to construct immediately the
epimorphism from $G$ to the regular permutation representation of $G/K$.
For the applications to {\bf quotpic} it is essential to be able to do this
(at least when $G/K$ is reasonably small), which is the principal reason why
we have kept the original generators $h(j,k)$ of $H$. As is described in
\cite{9}, it is often possible to simplify the presentation of $H$
considerably by applying further Tietze transformations that eliminate
redundant generators, or indeed to use one of the versions of the
Reidemeister-Schreier process that uses fewer generators of $H$ to begin with.
This would make it considerably more complicated for us to compute the
required permutation representation of $G/K$.  In the future, however, we
shall be considering including the option to produce smaller presentations of
$H$ for use in those cases when $G/K$ is too large to compute the regular
permutation representation. This would probably make it quicker and
less demanding on storage to compute at least the order of these large
quotients.

Returning to the Reidemeister-Schreier process, if we assume that $H$ is
normal in $G$, then the elements $g_l^{-1}h(j,k)g_l$ will lie in $H$.
By writing the $h(j,k)$ as words in the generators $g_i$, we can
regard these expressions as relators of $G$, and then express them as words in
the generators of $H$ as described above. In other words, we can compute the
conjugation action of the generators of $G$ on those of $H$. This, in turn,
enables us to compute the conjugation action of $G/H$ on $H/H^{(p)}$ as
an induced matrix group
over the field $K$ of order $p$. We can do the same for the action
of $G/H$ on any of the sections $\gamma_{i}^{p}(H)/\gamma_{i+1}^{p}(H)$ of the
lower exponent-$p$-central series of $H$ as computed by the $p$-quotient
algorithm. We can then use the submodule lattice package, which was
made available to us from Aachen, to compute the $K(G/H)$-submodules of
this induced module. These correspond precisely to the normal subgroups of
$G$ that lie between $H$ and $H^{(p)}$, or between $\gamma_{i}^{p}(H)$ and
$\gamma_{i+1}^{p}(H)$. Since the sublattice package can output generators
(as vector spaces) of each of the submodules found, there is no difficulty
in computing the regular permutation representation of the corresponding
quotient of $G$ (provided, of course, that it has small enough order).
In fact, since there may be a large number of these submodules, the user must
choose at the end of the calculation how many, if any, to plot on the screen.

We turn now to the lattice operations, intersection and join. Suppose that
we have permutation representations of $G$ on sets $\Omega_1$ and
$\Omega_2$, with kernels $M$ and $N$ respectively.
We replace $\Omega_2$ by a set that is
disjoint with $\Omega_1$, and then the natural induced action of $G$ on
$\Omega_1 \cup \Omega_2$ is a permutation representation of $G$ with kernel
$M \cap N$. We have provided a program {\bf convrep} which computes the
permutation representation of a permutation group $P$ in its action by right
multiplication on the right cosets of a subgroup $Q$ of $P$ and, by taking
$Q = 1$, we compute the regular permutation representation of the
quotient $G/M \cap N$ if its order is sufficiently small.
By comparing the order of this quotient with those of $G/M$ and $G/N$, we
can decide whether either of $M$ and $N$ is contained in the other, or
if $M = N$. In case of strict containment, an edge is plotted between the two
vertices if there is none there already. In case of equality, the two
vertices will be identified in the graph
(which might then result in the identification of further edges and vertices).

For the join
operation, we first calculate the action on $\Omega_2$ of the kernel $M$
of the action of $G$ on $\Omega_1$. To do this, we find bases
$b_1, b_2, \ldots ,b_r$ and $c_1,c_2, \ldots, c_s$ of the two permutation
representations. Then clearly $b_1, \ldots ,b_r, c_1, \ldots, c_s$
will be a base of the action of $G$ on $\Omega_1 \cup \Omega_2$, and a strong
generating set with respect to this base will contain generators of the
subgroup $M$ in its action on $\Omega_2$. Then we apply {\bf convrep} to
$G^{\Omega_2}$, using $M^{\Omega_2}$ as the subgroup.
The result will be the regular permutation representation of $G/MN$, which
is what we want.

Finally, the program {\bf orders}, which counts the number of elements
of each order in a regular permutation group $P$ on a set $\Omega$, is
straightforward. For each $i$ in $\Omega$, it
constructs the unique permutation in $P$ that takes  1  to $i$,  and computes
the length of the orbit of  1  in this permutation, to give its order.

\head 4. Examples and illustrations                % bold, centered; 
\endhead                             % don't type final punctuation 
We start with four examples which we were asked to analyze by Jeff Weeks.
They arose as fundamental groups of closed manifolds, and he wished to know
whether they were finite, to help determine whether or not the manifolds
were hyperbolic.
These were:
$$  G_1 = \langle a,b | ab^3a=b^2, a^3=b^5 \rangle,
 G_2 = \langle a,b | a=b^3a^2b^3, a^3=b^7 \rangle,$$
$$  G_3 = \langle a,b | a=b^3ab^3, ba^2(b^{-2}a^2ba^2)^2=1 \rangle,
G_4 = \langle a,b | a^2ba^2=b^3, a^3b^4=1 \rangle.$$

Using {\bf quotpic}, we managed to
determine the precise structures of $G_1$ and $G_4$,
both of which turned out to be finite. In fact $G_1 \cong
SL(2,5) \times C_{13}$ (where  $C_{13}$ is the cyclic group of order 13),
and $G_4 \cong \hat{S}_4 \times C_{11}$, where
$\hat{S}_4$ is the 2-fold covering group of $S_4$ that is {\it not}
isomorphic to $GL(2,3)$. (The `orders' option to count the number of
elements of each order in a group was used to distinguish between the
two nonisomorphic 2-fold covering groups of $S_4$.) Having obtained
the above as quotients of $G_1$ and $G_4$, we succeeded in recognizing
these quotients as the full group in both cases, by calculating
presentations of the relevant normal subgroups and finding them to be
trivial presentations. (Of course, the orders of these groups can also
be determined very quickly by a Todd-Coxeter coset enumeration.) 

On the other
hand, $G_2$ and $G_3$ both turned out to be infinite with very many finite
quotients. For example, $G_2$ has
a map onto $PSL(2,7)$ of which the kernel has $C_{100} \times C_\infty^6$ as
maximal abelian quotient, and $G_3$ has a map onto $A_5$  of which the kernel
has $C_2^2 \times C_\infty^{29}$ as maximal abelian quotient.

As an illustration, Fig. 1 shows two views of what the user sees when
carrying out this computation in $G_1$. Although it cannot be seen in these
pictures, selecting the `ab. invariants' menu option shows us that the maximal
abelian quotient of $G_1$ itself has order 13. This proves that $G_1$ has
the 2-fold cover of $A_5$ (i.e. $SL(2,5)$) as a direct factor, rather than
$A_5 \times C_2$, and so the structure of the group is unambiguous. The
vertex representing the quotient of order 1560 is triangular (or yellow) in
the first picture. The menu item `conv. to reg. rep.' was then selected to
compute the regular permutation representation of this quotient, which
automatically converted the vertex to circular (green), and then the full
menu of operations became available. It was then possible to select the
menu option `rep. kernel', which computed a presentation of the normal
subgroup corresponding to this vertex, and this presentation turned out to be
trivial.

\topinsert
\special{psfile=quotpicfig1}
\vskip 67mm
\botcaption {Figure 1} \endcaption
\endinsert

Our next illustrative example is the Hurwitz group
$G = \langle x,y | x^2=y^3=(xy)^7=1 \rangle$.
In Fig. 2, we have found the (unique) epimorphisms of $G$
onto the simple groups $PSL(2,7)$ and $PSL(2,8)$. The maximal abelian quotients
of their kernels were $C_\infty^6$ and $C_\infty^{14}$ respectively. We
have then plotted the maximal elementary abelian  $p$-quotients of these
kernels, for $p=2$ and 3 in the first case and $p=2$ in the second, and used
the `intermediate quotient' (`Meataxe') option to find and plot
all intermediate quotients. The original edges representing the full
elementary abelian quotients have been removed where possible, in order
to clarify the diagram. Notice that many of the vertices are octagonal (red).
Finally, in order to illustrate the parameters
section of the menu, we have taken the picture while it was in the process
of calculating the maximal elementary abelian 5-quotient of one of the
normal subgroups.

\topinsert
\special{psfile=quotpicfig2}
\vskip 115mm
\botcaption {Figure 2} \endcaption
\endinsert

The following  interesting example was proposed some years ago by Heineken.
$G = \langle x,y,z | [[x,[x,y]]=z, [[y,[y,z]]=x, [z,[z,x]]=y  \rangle$.
Neub\"{u}ser in Aachen discovered that it had an epimorphism onto $A_5$
of which the kernel had a quotient of class 5 and order $2^{24}$. No other
quotients of this group are known. In Fig. 3, we have applied the
$p$-quotient algorithm to the kernel of the map onto $A_5$ to obtain the
terms in the descending 2-central series. These occur down the main diagonal
in the picture. In addition, we have found those intermediate quotients
that lie between the successive terms of the series, and plotted these to
the left of the main diagonal. In particular, we notice that the two
layers of order $2^5$ have different submodule structures. (In fact, there
are two nonisomorphic $A_5$-modules of order $2^4$ involved in this sequence,
and the `Meataxe' package does distinguish between them, but we have not yet
made it possible to display this information.) Finally, we have opened the
global menu.

\topinsert
\special{psfile=quotpicfig3}
\vskip 123mm
\botcaption {Figure 3} \endcaption
\endinsert

As a final example, we should like to mention a group that was
produced at the DIMACS conference itself by Leonard Soicher and Jon Hall.
This has 5 involutory generators and far too many relations to write
down here. (The interested reader could obtain the presentation from
us, or from them.) The conjecture is that it defines the largest
3-transposition group generated by 5 transpositions, such that no two
transpositions commute. If so, then it should be an extension of a large 
3-group by a group of order 2. We discovered, by using the $p$-quotient
algorithm from within {\bf quotpic}, that it has a quotient that is
an extension of a group of order $3^{49}$ (and exponent-3-class 6)
by a group of order 2, and that this is the largest quotient of this form.
We found no other finite quotients of this group. This information is
therefore consistent with the conjecture.

\head 5. Future development                % bold, centered; 
\endhead                             % don't type final punctuation 
{\bf quotpic} is still very much under development.
We have plans to
\roster
\item"(a)" include a low index subgroup algorithm (see Section 6 of \cite{9}).
This would give an alternative approach to that of {\bf permim}, and would 
allow us to hold several permutation representations of the same quotient. 
\item"(b)" allow the possibility of applying the Reidemeister-Schreier related
programs to the point stabilizers in non-regular permutation representations
(which are in general not normal subgroups). The examination of non-normal
subgroups has been used in the past both to prove infiniteness (see \cite{6}), 
and to distinguish between non-isomorphic groups (see \cite{2}). 
\item"(c)" replace our own implementation of the Canberra $p$-quotient program
by the much more efficient implementation of Eamonn O'Brien and Mike
Newman. 
\item"(d)" include the general nilpotent quotient algorithm, as implemented by
Werner Nickel. This will enable us to gather information also about infinite
nilpotent sections.
\item"(e)" make it possible for {\bf quotpic} to look for all maps onto 
{\it any} permutation group on which the user has supplied information,
rather than restrict to a specific list of permutation groups held by the 
program.  
\endroster

We are keen to receive comments from users about our program, and will 
be very pleased to give copies (via magnetic tape or FTP)
to anyone who asks for it.
We invite suggestions for improvements, and would like any bugs, or
odd behaviour to be reported; ideally a bug report should be accompanied by a
copy of the presentation input to the program, together with the log file
created by {\bf quotpic}.

The files that can be collected via anonymous FTP are compressed UNIX tarred
files named isom{\it i}.tar.Z for $i = 1,2,3, \ldots $. They can be found
either in the directory pub/local/nser on the machine tuda.ncl.ac.uk or in the
directory pub/sources/geom on the machine amazon.warwick.ac.uk.

\Refs

\ref \no 1
\by G. Havas
\paper A Reidemeister-Schreier program
\inbook  Proceedings of the
Second International Conference on the Theory of Groups, Canberra 1973,
Lecture Notes in Math., vol. 372,
\ed M.F. Newman
\publ   Springer-Verlag \publaddr Berlin and New York
\yr 1974 \pages 347--356
\endref
\ref \no 2
\by G. Havas and L.G.Kov\'{a}cs
\paper Distinguishing eleven crossing knots
\inbook Computational Group Theory
\ed M. Atkinson
\publ Academic Press \publaddr London
\yr 1984
\pages 367--73
\endref
\ref \no 3
\by G. Havas and M.F. Newman
\paper Applications of computers to questions like those of Burnside
\inbook Burnside Groups, Lecture Notes in Math., vol. 806,
\ed J.L. Mennicke
\publ   Springer-Verlag \publaddr Berlin and New York
\yr 1980 \pages 211--30
\endref
\ref \no 4
\by  G. Havas and L. Sterling
\paper Integer matrices and abelian groups
\inbook Symbolic and Algebraic Computation,
Lecture Notes in Computer Science, vol. 72,
\ed E.W. Ng
\publ   Springer-Verlag \publaddr Berlin and New York
\yr 1979 \pages 431--451
\endref
\ref \no 5
\by D. F. Holt and W. Plesken
\book Perfect Groups
\publ Oxford University Press
\yr 1989
\endref
\ref \no 6
\by D. F. Holt and W. Plesken
\paper A cohomological criterion for a finitely presented group to be 
infinite
\toappear
\endref
\ref \no 7
\by D. F. Holt and Sarah Rees
\paper Testing for isomorphism between finitely presented groups
\inbook Groups, Combinatorics and Geometry, LMS Lecture Note Series 165,
\eds M.W. Liebeck and J. Saxl
\publ Cambridge University Press
\yr 1992
\pages 459--475
\endref
\ref \no 8
\by  K. Lux, J. M\"uller and M. Ringe
\paper Peakword condensation and submodule lattices,
an application of The Meat-Axe
\toappear
\endref
\ref \no 9
\by  J. Neub\"{u}ser
\paper An elementary introduction to coset table methods in
computational group theory
\inbook  Groups - St Andrews 1981, LMS Lecture Note Series 71,
\eds C. M.  Campbell and E. F. Robertson
\publ Cambridge University Press
\yr 1982
\pages 1--45
\endref
\ref \no 10
\by R. Parker
\paper The computer calculation of modular characters.  (The Meat-Axe)
\inbook Computational Group Theory
\ed M. Atkinson
\publ Academic Press \publaddr London
\yr 1984
\pages 267--74
\endref

\endRefs

\enddocument

%-----------------------------------------------------------------------------
% End of fpgroups.tex
%-----------------------------------------------------------------------------

